\documentclass{article}
\usepackage{graphicx}
\usepackage{fullpage}
\usepackage{amsmath}
\usepackage{amsthm}
\usepackage{xspace}
%
\graphicspath{{figures/}}
%
\theoremstyle{definition}
\newtheorem{definition}{Definition}
%
\newcommand{\model}{\ensuremath{\mathrm{M}}\xspace}
\newcommand{\sd}{\ensuremath{\mathrm{SD}}\xspace}
\newcommand{\comps}{\ensuremath{\mathrm{COMPS}}\xspace}
\newcommand{\pr}[1]{\ensuremath{\mbox{Pr}(#1)}}

%
\title{Stochastic and Quantum Algorithms for Model-Based Diagnosis of Digital Circuits}
%
% Title for submission to Science: Quantum Speedup = Diagnostic Optimality
%
\author{Alejandro Perdomo-Ortiz, Alexander Feldman, Asier Ozaeta, and Johan de Kleer}
%
\begin{document}
\maketitle
\section{Introduction}

\section{Definitions}
%
This section gives the basic definitions used in the algorithms we
design and analyze.
%
\begin{definition}[Basis]
  %
  A basis $\mathcal{B}$ is a set of propositional formulas $\{f_1 :
  o_1 \leftrightarrow F_1, f_2 : o_2 \leftrightarrow F_2, \cdots f_n :
  o_n \leftrightarrow F_n\}$ where $o_1, o_2, \ldots, o_n$ are outputs
  and $F_1, F_2, \ldots, F_n$ are propositional subformulas
  implementing Boolean operations.
  %
\end{definition}
\noindent
%
We form a base from the common logic-gate types: AND, OR, NAND, NOR,
XOR, inverter, and buffer. The AND, OR, NAND, and NOR gates may have
more than two inputs. Figure~\ref{fig:basis} shows these logic gates
pictographically. In text and formulas we use the common propositional
logic connectives: $\wedge$, $\vee$, and $\neg$, $\rightarrow$, and
$\leftrightarrow$. The last two (implication and equivalence) will not
appear in the circuits we are trying to diagnose (and are not part of
the basis) but we use them in theory and algorithms. The basis element
for the AND gate, for example, becomes $o \leftrightarrow i_1 \wedge
i_2$. The semantics of all this is usual and is explained in any
introductory logic or VLSI design book.
%
\begin{figure}[htb]
\centering
\includegraphics{basis.mps}
\caption{Pictographic representation of common logic gates\label{fig:basis}}
\end{figure}
\par
%
Logic circuits are designed by connecting gates from the basis. A
well-formed logic-circuit design is a Direct Acyclic Graph (DAG),
there are no hanging edges which would be a violation of the common
definition for a DAG.
%
\begin{definition}[Boolean Circuit]
  %
  A Boolean circuit $\model = \{V, E\}$ is a DAG in which each edge
  $e \in E$ is a variable and each node $v \in V$ is either a
  logic-gate or a primary input or a primary output.
  %
\end{definition}
%
We have to introduce special primary input and primary output nodes,
which we do not normally draw in a circuit diagram. Boolean circuits
normally perform some useful function such as adding or multiplying
numbers, detecting or correcting errors, etc. Combined with memory
elements (flip-flops) they become the basis of most modern computing
(except quantum).
\par
Throughout this paper we will use the running example in
figure~\ref{fig:adder} as a system to be diagnosed. Its function is to
sum-up two two-bit numbers. The result is a two-bit sum and a
carry-bit.
%
\begin{figure}[htb]
\centering
\includegraphics{adder_2.mps}
\caption{$2$-bit adder\label{fig:adder}}
\end{figure}
%
\begin{definition}[Fault-Augmented Boolean Circuit]
\end{definition}

\begin{definition}[Observation]
  %
  An observation $\alpha$ is an assignment to some or all primary
  inputs and primary outputs of a Boolean circuit \model.
  %
\end{definition}

\begin{definition}[Fault-Injection]
  %
  A fault-injection $\phi$ is an assignment to all fault variables of
  a fault-augmented Boolean circuit \sd.
  %
\end{definition}

\begin{definition}[Diagnosis]
  %
  Given a fault-augmented Boolean circuit \sd and an observation
  $\alpha$, a diagnosis $\omega$ is defined as an assignment to all
  fault variables in \sd such that $\sd \wedge \alpha \wedge \omega
  \not\models \perp$.
  %
\end{definition}

\begin{definition}[Boolean Circuit Health Estimation]
  %
  Given a fault-augmented Boolean circuit \sd, a Boolean circuit
  fault estimation $H$ is defined as the probability of each fault
  variable being $\perp$.
  %
\end{definition}

\begin{definition}[Isolation Accuracy]
  %
  Given a fault injection $\phi$ and a fault estimation $H$ belonging
  to the same fault-augmented Boolean circuit \sd, the isolation
  accuracy $M_{ia}(\phi, H)$ is defined as:
  %
  \begin{eqnarray}
    M_{ia}(\phi, H) = \sum_{f \in \comps}{\pr{H(f) = \top} - \pr{\phi(f) = \top}}
  \end{eqnarray}
  %
  where $\pr{\phi(f) = \top} = 1$ if $\phi(f) = \top$ and
  $\pr{\phi(f) = \top} = 0$, otherwise.
  %
\end{definition}
%
\section{Fundamentals of Circuit Diagnostics}
\section{Circuit Diagnosis and Polynomial Minimization}
\section{Simulated Annealing}
\section{Quantum Annealing}
\section{Experiments}
%
\begin{figure}[htb]
\centering
\includegraphics[scale=0.5]{multiplier.mps}
\caption{$n$-bit parallel multiplier\label{fig:multiplier}}
\end{figure}
%
\section{Related Work}
\section{Conclusions}

\end{document}
