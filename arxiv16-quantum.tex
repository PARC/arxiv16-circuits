\documentclass{article}
\usepackage{graphicx}
\usepackage{fullpage}
\usepackage{amsthm}
%
\graphicspath{{figures/}}
%
\theoremstyle{definition}
\newtheorem{definition}{Definition}
%
\newcommand{\model}{\ensuremath{\mathrm{M}}\xspace}
\newcommand{\sd}{\ensuremath{\mathrm{SD}}\xspace}
\newcommand{\comps}{\ensuremath{\mathrm{COMPS}}\xspace}
\newcommand{\pr}[1]{\ensuremath{\mbox{Pr}(#1)}}

%
\title{Stochastic and Quantum Algorithms for Model-Based Diagnosis of Digital Circuits}
%
% Title for submission to Science: Quantum Speedup = Diagnostic Optimality
%
\author{Alejandro Perdomo-Ortiz, Alexander Feldman, Asier Ozaeta, and Johan de Kleer}
%
\begin{document}
\maketitle
\section{Introduction}
\section{Running Example}
gen-adder 2
\section{Definitions}

\begin{definition}[model/circuit]
\end{definition}

\begin{definition}[fault injection]
\end{definition}

\begin{definition}[observation]
\end{definition}

\begin{definition}[diagnosis]
\end{definition}

\begin{definition}[? health state of a gate (binomial pdf)]
\end{definition}

\begin{definition}[? health state of a circuit (combination of all binomal pdfs)]
\end{definition}

\begin{definition}[most informed (health state/pdf)]
\end{definition}

\begin{definition}[metric - diff between most informed pdf and health state pdf]
\end{definition}

\section{Fundamentals of Circuit Diagnostics}
\section{Circuit Diagnosis and Polynomial Minimization}
\section{Simulated Annealing}
\section{Quantum Annealing}
\section{Experiments}
%
\begin{figure}[htb]
\centering
\includegraphics[scale=0.4]{multiplier.mps}
\caption{$n$-bit parallel multiplier\label{fig:multiplier}}
\end{figure}
%
\section{Related Work}
\section{Conclusions}

\end{document}
